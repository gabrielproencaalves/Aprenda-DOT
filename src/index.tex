\documentclass[a4paper,12pt]{article}
\usepackage[utf8]{inputenc}
\usepackage[T1]{fontenc}
\usepackage[scaled]{helvet}
\renewcommand{\familydefault}{\sfdefault}
\usepackage{lmodern}
\usepackage[portuguese]{babel}
\usepackage{graphicx}
\usepackage{url}
\usepackage{indentfirst}
\pagestyle{plain}

\begin{document}
  \title{\vspace{-1.5in} Guia introdutório de \emph{DOT}}
  \author{Gabriel P. Alves}
  \date{}
  \maketitle

  \section{Prólogo}

    \subsection{O que é \emph{DOT}?}
      DOT é uma linguagem de marcação
      utilizada para construir documentos
      contendo redes de informações,
      informações estruturadas, grafos
      descritivos e diagramas.

      Uma das principais características da
      linguagem é a sua simplicidade. Por
      meio dela, podemos escrever
      documentos de forma prática e
      personalizável, sem gastarmos muito
      tempo com documentações exaustivas.

      Além disso, ela facilita o
      versionamento de documentos, uma vez
      que qualquer código \emph{dot} pode ser
      armazenado em um arquivo de
      texto simples.

      Um arquivo \emph{dot} é lido e
      interpretado por um conjunto de
      ferramentas chamado \emph{Graphviz},
      das quais podem gerar documentos de
      diversos formatos. Esses utilitários
      consomem pouquíssimos recursos e são
      capazes de manipular grafos complexos
      em pouco tempo.

    \subsection{Empregabilidade}
      % Atualmente o dot pode...
      Atualmente, a linguagem \emph{dot} e
      o seu conjunto de interpretadores
      \emph{Graphviz} possuem importantes
      aplicações no planejamento de redes,
      bioinformática, engenharia de software,
      desenvolvimento web com banco de dados,
      machine learning e outras áreas técnicas.

      Isso nos mostra a importância do estudo, e do
      domínio, destas ferramentas para qualquer
      tipo de profissional relacionado aos ramos da
      tecnologia.

    \break

  \section{Os primeiros passos}

    \subsection{Elementos}
      Antes de construirmos os nossos primeiros diagramas,
      precisamos entender quais são os elementos básicos
      que compõem um documento DOT.

      \subsubsection{\emph{graph}}
        % Os grafos...

        \begin{itemize}
          % Existem dois tipos...

          \item{\emph{graph}}
            % grafos não direcionados...

          \item{\emph{digraph}}
            % grafos direcionados...

        \end{itemize}

      \subsubsection{\emph{node}}
        % Os nodes...

      \subsubsection{\emph{edge}}
        % Os edges...

      % inserir aqui, uma imagem
      % explicativa sobre os assuntos
      % anteriores

  \section{Estruturando o documento}
    % Até o momento...

    \subsection{\emph{subgraph}}
      % O subgraph...

    \subsection{\emph{rank}}
      % O atributo rank...

    \subsection{\emph{rankdir}}
      % O atributo rankdir...
    \break

  \section{Personalizando o documento}
    % A partir de agora...

    \subsection{Atributos}
      % Os atributos são...

      \begin{itemize}
        \item{\emph{graph}}
          % Os atributos dos grafos...

        \item{\emph{node}}
          % Os atributos dos nós...

        \item{\emph{edge}}
          % os atributos das arestas...
      \end{itemize}

  \section{Entendendo a documentação}
    % Primeiramente, eu gostaria de lhe dar os
    % parabéns por ter aprendido os conceitos e
    % chegado até aqui... (draft)

\end{document}
